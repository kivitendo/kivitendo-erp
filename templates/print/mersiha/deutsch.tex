%%%%%%%%%%%%%%%%%%%%%%%%%%%%%%%%%%%%%%%%%%%%%
%standardphrasen und schnipsel in deutsch     %
%dient als vorlage für alle anderen sprachen  %
%%%%%%%%%%%%%%%%%%%%%%%%%%%%%%%%%%%%%%%%%%%%%%%


\newcommand{\anrede} {Sehr geehrte Damen und Herren,}
\newcommand{\anredefrau} {Sehr geehrte Frau}
\newcommand{\anredeherr} {Sehr geehrter Herr}


\newcommand{\nr} {Nr.}
\newcommand{\datum} {Datum}
\newcommand{\kundennummer} {Kunden-Nr.}
\newcommand{\ansprechpartner} {Ansprechpartner}
\newcommand{\bearbeiter} {Bearbeiter}
\newcommand{\gruesse} {Mit freundlichen Grüßen}
\newcommand{\vom} {vom}
\newcommand{\von} {von}
\newcommand{\seite} {Seite}
\newcommand{\uebertrag} {Übertrag}


\newcommand{\position} {Pos.}
\newcommand{\artikelnummer} {Art.-Nr.}
\newcommand{\bild} {Bild}
\newcommand{\keinbild} {kein Bild}
\newcommand{\menge} {Menge}
\newcommand{\bezeichnung} {Bezeichnung}
\newcommand{\seriennummer}{Seriennummer}
\newcommand{\ean}{EAN}
\newcommand{\projektnummer}{Projektnummer}
\newcommand{\charge}{Charge}
\newcommand{\mhd}{MHD}
\newcommand{\einzelpreis} {E-Preis}
\newcommand{\gesamtpreis} {G-Preis}
\newcommand{\nettobetrag} {Nettobetrag}
\newcommand{\schlussbetrag} {Gesamtbetrag}
\newcommand{\optional} {optional}
\newcommand{\vorgangsbez} {Vorgangsbez.}

\newcommand{\weiteraufnaechsterseite} {weiter auf der nächsten Seite \ldots}

\newcommand{\zahlung} {Zahlungsbedingungen}
\newcommand{\lieferung} {Lieferbedingungen}
\newcommand{\textTelefon} {Tel.}
\newcommand{\textEmail} {E-Mail}
\newcommand{\textFax} {Fax}

% angebot (sales_quotion)
\newcommand{\angebot} {Angebot}
\newcommand{\angebotsformel} {gerne unterbreiten wir Ihnen folgendes Angebot:}
\newcommand{\angebotdanke} {Wir danken für Ihre Anfrage und hoffen, Ihnen hiermit ein interessantes Angebot gemacht zu haben.}
\newcommand{\angebotgueltig} {Das Angebot ist freibleibend gültig bis zum}% Danach wird das Datum eingefügt, falls das grammatisch nicht funktionieren sollte müssen wir eine ausnahme für die sprache definieren
\newcommand{\angebotfragen} {Sollten Sie noch Fragen oder Änderungswünsche haben, können Sie uns gerne jederzeit unter den unten genannten Telefonnummern oder E-Mail-Adressen kontaktieren.}
\newcommand{\angebotagb} {Bei der Durchführung des Auftrags gelten unsere AGB, die wir Ihnen gerne zuschicken.}
\newcommand{\auftragerteilt}{Auftrag erteilt:}
\newcommand{\angebotortdatum}{Wir nehmen das vorstehende Angebot an.}
\newcommand{\abweichendeLieferadresse}{abweichende Lieferadresse}

% angebotseingang (purchase_quotation_intake)
\newcommand{\angebotseingang} {Angebotseingang}
\newcommand{\angebotseingangformel} {vielen Dank für Ihr Angebot. Wir haben es als Angebotseingang erfasst und werden es prüfen.}

% auftragseingang (sales_order_intake)
\newcommand{\auftragseingang} {Auftragseingang}
\newcommand{\auftragseingangnummer} {Auftragseingang Nr.}
\newcommand{\auftragseingangformel} {vielen Dank für Ihren Auftrag, wir haben Ihn als Auftragseingang erfasst.\\Sie erhalten innerhalb von drei Tagen unsere Auftragsbestätigung.}

% auftragbestätigung (sales_order)
\newcommand{\auftragsbestaetigung} {Auftragsbestätigung}
\newcommand{\auftragsnummer} {Auftrags-Nr.}
\newcommand{\ihreBestellnummer} {Ihre Bestellnummer}
\newcommand{\auftragsformel} {hiermit bestätigen wir Ihnen folgende Bestellpositionen:}
\newcommand{\lieferungErfolgtAm} {Die Lieferung erfolgt am} %Danach wird das Datum eingefügt, falls das grammatisch nicht funktionieren sollte müssen wir eine ausnahme für die sprache definieren
\newcommand{\auftragpruefen} {Bitte kontrollieren Sie alle Positionen auf Übereinstimmung mit Ihrer Bestellung! Teilen Sie Abweichungen innerhalb von 3 Tagen mit!}
\newcommand{\proformarechnung} {Proforma Rechnung}
\newcommand{\nurort} {Ort}
\newcommand{\den} {den}
\newcommand{\unterschrift} {Unterschrift}
\newcommand{\stempel} {ggf. Stempel}

% lieferschein (sales_delivery_order)
\newcommand{\lieferschein} {Lieferschein}
\newcommand{\retourenlieferschein} {Retouren-Lieferschein}

% verkaufsreklamationen (sales_reclamation)
\newcommand{\vkreklamation}{VK-Reklamation}
\newcommand{\vkreklamationsformel} {hiermit bestätigen wir Ihnen den Eingang einer Ihrer Reklamation mit folgenden Positionen:}

% einkaufs-reklamationen (purchase_reclamation)
\newcommand{\ekreklamation}{EK-Reklamation}
\newcommand{\ekreklamationsformel} {folgende Positionen werden reklamiert:}

\newcommand{\grund}{Grund}

% rechnung (invoice)
\newcommand{\rechnung} {Rechnung}
\newcommand{\rechnungskopie} {Rechnungskopie}
\newcommand{\rechnungsdatum} {Rechnungsdatum}
\newcommand{\ihrebestellung} {Ihre Bestellung}
\newcommand{\lieferdatum} {Lieferdatum}
\newcommand{\rechnungsformel} {für unsere Leistungen erlauben wir uns, folgende Positionen in Rechnung zu stellen:}
\newcommand{\zwischensumme} {Zwischensumme}
\newcommand{\leistungsdatumGleichRechnungsdatum} {Das Leistungsdatum entspricht, soweit nicht anders angegeben, dem Rechnungsdatum.}
\newcommand{\leistungsdatum} {Leistungsdatum}
\newcommand{\unserebankverbindung} {Unsere Bankverbindung}
\newcommand{\textKontonummer} {Kontonummer:}
\newcommand{\textBank} {bei der}
\newcommand{\textBankleitzahl} {BLZ:}
\newcommand{\textBic} {BIC:}
\newcommand{\textIban} {IBAN:}
\newcommand{\unsereustid} {Unsere USt-Identifikationsnummer lautet}
\newcommand{\ihreustid} {Ihre USt-Identifikationsnummer:}
\newcommand{\steuerfreiEU} {Sonstige Leistungen Steuerschuldnerschaft des Leistungsempfängers. Reverse Charge}
\newcommand{\steuerfreiAUS} {Steuerfreie Lieferung ins außereuropäische Ausland.}

\newcommand{\textUstid} {USt-IdNr.:}
\newcommand{\stornorechnung} {Stornorechnung}

% anzahlungsrechnung (invoice_for_advance_payment)
\newcommand{\anzahlungsrechnung} {Anzahlungsrechnung}
\newcommand{\schlussrechnung} {Schlussrechnung}
\newcommand{\ust} {USt}
\newcommand{\abzueglichAnzahlungsrechnungen} {Abzüglich folgender Anzahlungsrechnungen}
\newcommand{\rechnungsbetrag} {Rechnungsbetrag}

% gutschrift (credit_note)
\newcommand{\gutschrift} {Gutschrift}
\newcommand{\fuerRechnung} {für Rechnung}
\newcommand{\gutschriftformel} {wir erlauben uns, Ihnen folgende Positionen gutzuschreiben:}

% sammelrechnung (statement)
\newcommand{\sammelrechnung} {Sammelrechnung}
\newcommand{\sammelrechnungsformel} {bitte nehmen Sie zur Kenntnis, dass folgende Rechnungen unbeglichen sind:}
\newcommand{\faellig} {Fälligkeit}
\newcommand{\aktuell} {aktuell}
\newcommand{\asDreissig} {30}
\newcommand{\asSechzig} {60}
\newcommand{\asNeunzig} {90+}

% zahlungserinnerung (Mahnung)
\newcommand{\mahnung} {Zahlungserinnerung}
\newcommand{\mahnungsformel} {man kann seine Augen nicht überall haben - offensichtlich haben Sie übersehen, die folgenden Rechnungen zu begleichen:}
\newcommand{\beruecksichtigtBis} {Zahlungseingänge sind nur berücksichtigt bis zum}
\newcommand{\schonGezahlt} {Sollten Sie zwischenzeitlich bezahlt haben, betrachten Sie diese Zahlungserinnerung bitte als gegenstandslos.}

% zahlungserinnerung_invoice (Rechnung zur Mahnung)
\newcommand{\mahnungsrechnungsformel} {hiermit stellen wir Ihnen zu o.g. \mahnung \ folgende Posten in Rechnung:}
\newcommand{\posten} {Posten}
\newcommand{\betrag} {Betrag}
\newcommand{\bitteZahlenBis} {Bitte begleichen Sie diese Forderung bis zum}

% anfrage (request_quotion)
\newcommand{\anfrage} {Anfrage}
\newcommand{\anfrageformel} {bitte nennen Sie uns für folgende Artikel Preis und Liefertermin:}
\newcommand{\anfrageBenoetigtBis} {Wir benötigen die Lieferung bis zum}  %Danach wird das Datum eingefügt, falls das grammatisch nicht funktionieren sollte müssen wir eine ausnahme für die sprache definieren
\newcommand{\anfragedanke} {Im Voraus besten Dank für Ihre Bemühungen.}

% bestellung/auftrag (purchase_order)
\newcommand{\bestellung} {Bestellung}
\newcommand{\unsereBestellnummer} {Unsere Bestellnummer}
\newcommand{\bestellformel} {hiermit bestellen wir verbindlich folgende Positionen:}

% einkaufslieferschein (purchase_delivery_order)
\newcommand{\einkaufslieferschein} {Eingangslieferschein}
\newcommand{\beistelllieferschein} {Beistell-Lieferschein}

% Brief/letter
\newcommand{\ihrzeichen}{Ihr Zeichen}
\newcommand{\betreff}{Betreff}

%check
\newcommand*{\ausgestellt}{Ausgestellt}
\newcommand*{\verrechnet}{Verrechnet}

%lagerliste/bin_list
\newcommand*{\lagerliste}{Lagerliste}
\newcommand*{\lieferanschrift}{Lieferanschrift}
\newcommand*{\bestellnummer}{Bestellnummer}
\newcommand*{\kontakt}{Kontakt}
\newcommand*{\lager}{Lager}
\newcommand*{\lagerplatz}{Lagerplatz}
\newcommand*{\lieferungMit}{Lieferung mit}
\newcommand*{\nummer}{Nummer}
\newcommand*{\beschreibung}{Beschreibung}
\newcommand*{\erh}{Erh}

%sammelliste
\newcommand*{\sammelliste}{Sammelliste}
\newcommand*{\lagerausgang}{Lagerausgang}

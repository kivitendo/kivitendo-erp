%% insettings.tex
%% Copyright 2019 Marei Peischl
\ProvidesFile{insettings.tex}[2019/12/22 Konfigurationsdatei kivitendo ERP]
% Sprachüberprüfung
\RequirePackage[english, ngerman]{babel}
\Ifstr{\lxlangcode}{EN}{
	\makeatletter
	\main@language{english}
	\makeatother
	%%%%%%%%%%%%%%%%%%%%%%%%%%%%%%%%%%%%%%%%%%%%%%%
%standardphrasen und schnipsel in deutsch     %
%dient als vorlage für alle anderen sprachen  %
%%%%%%%%%%%%%%%%%%%%%%%%%%%%%%%%%%%%%%%%%%%%%%%


\newcommand{\anrede} {Dear Sirs,}
\newcommand{\anredefrau} {Dear Ms}
\newcommand{\anredeherr} {Dear Mr}


\newcommand{\nr} {No.}
\newcommand{\datum} {Date}
\newcommand{\kundennummer} {Customer-No.}
\newcommand{\ansprechpartner} {Contact person}
\newcommand{\bearbeiter} {Employee}
\newcommand{\gruesse} {Sincerely yours, }
\newcommand{\vom} {from}
\newcommand{\von} {from}
\newcommand{\seite} {page}
\newcommand{\uebertrag} {amount carried over}


\newcommand{\position} {Pos.}
\newcommand{\artikelnummer} {Part No.}
\newcommand{\bild} {Picture}
\newcommand{\keinbild} {n/a}
\newcommand{\menge} {Qty}
\newcommand{\bezeichnung} {Description}
\newcommand{\seriennummer}{Serial No.}
\newcommand{\ean}{EAN}
\newcommand{\projektnummer}{Projekt No.}
\newcommand{\charge}{Charge}
\newcommand{\mhd}{Best before}
\newcommand{\einzelpreis} {Price}
\newcommand{\gesamtpreis} {Amount}
\newcommand{\nettobetrag} {Net amount}
\newcommand{\schlussbetrag} {Total}

\newcommand{\weiteraufnaechsterseite} {to be continued on next page  ..}

\newcommand{\zahlung} {Payment terms:}


% angebot (sales_quotion)
\newcommand{\angebot} {Quotation}
\newcommand{\angebotsformel} {we are pleased to make the following offer:}
\newcommand{\angebotdanke} {We thank you for your request and look forward to receiving your order.}
\newcommand{\angebotgueltig} {This offer is valid until}		%Danach wird das Datum eingefügt, falls das grammatisch nicht funktionieren sollte müssen wir eine ausnahme für die sprache definieren
\newcommand{\angebotfragen} {If you have any questions do not hesitate to conatct us.}
\newcommand{\angebotagb} {Our general terms and conditions (AGB) apply. We will send them to you on request.}


% auftragbestätigung (sales_order)
\newcommand{\auftragsbestaetigung} {Order}
\newcommand{\auftragsnummer} {Order No.}
\newcommand{\ihreBestellnummer} {Your reference no.}
\newcommand{\auftragsformel} {We hereby confirm your order for the following items:}
\newcommand{\lieferungErfolgtAm} {Your items will be delivered on:} %Danach wird das Datum eingefügt, falls das grammatisch nicht funktionieren sollte müssen wir eine ausnahme für die sprache definieren
\newcommand{\auftragpruefen} {Please check that all items correspond to your order. Please tell us of any deviations within 3 days.}

% lieferschein (sales_delivery_order)
\newcommand{\lieferschein} {Delivery order}

% rechnung (invoice)
\newcommand{\rechnung} {Invoice}
\newcommand{\rechnungsdatum} {Invoice date}
\newcommand{\ihrebestellung} {Your order}
\newcommand{\lieferdatum} {Delivery date}
\newcommand{\rechnungsformel} {für unsere Leistungen erlauben wir uns, folgende Positionen in Rechnung zu stellen:}
\newcommand{\zwischensumme} {Subtotal}
\newcommand{\leistungsdatumGleichRechnungsdatum} {Das Leistungsdatum entspricht, soweit nicht anders angegeben, dem Rechnungsdatum.}
\newcommand{\unserebankverbindung} {Our bank details}
\newcommand{\textKontonummer} {Accout no.:}
\newcommand{\textBank} {bei der}
\newcommand{\textBankleitzahl} {BLZ:}
\newcommand{\textBic} {BIC:}
\newcommand{\textIban} {IBAN:}
\newcommand{\unsereustid} {Unsere USt-Identifikationsnummer lautet}
\newcommand{\ihreustid} {Ihre USt-Identifikationsnummer:}
\newcommand{\steuerfreiEU} {Steuerfreie, innergemeinschaftliche Lieferung.}
\newcommand{\steuerfreiAUS} {Steuerfreie Lieferung ins außereuropäische Ausland.}

\newcommand{\textUstid} {UStId:}

% gutschrift (credit_note)
\newcommand{\gutschrift} {Credit note}
\newcommand{\fuerRechnung} {for invoice}
\newcommand{\gutschriftformel} {we credit you with the following items:}

% sammelrechnung (statement)
\newcommand{\sammelrechnung} {Statement}
\newcommand{\sammelrechnungsformel} {please note that following invoices are outstanding:}
\newcommand{\faellig} {Due}
\newcommand{\aktuell} {Current}
\newcommand{\asDreissig} {30}
\newcommand{\asSechzig} {60}
\newcommand{\asNeunzig} {90+}

% zahlungserinnerung (Mahnung)
\newcommand{\mahnung} {Zahlungserinnerung}
\newcommand{\mahnungsformel} {man kann seine Augen nicht überall haben - offensichtlich haben Sie übersehen, die folgenden Rechnungen zu begleichen:}
\newcommand{\beruecksichtigtBis} {Zahlungseingänge sind nur berücksichtigt bis zum}
\newcommand{\schonGezhalt} {Sollten Sie zwischenzeitlich bezahlt haben, betrachten Sie diese Zahlungserinnerung bitte als gegenstandslos.}

% zahlungserinnerung_invoice (Rechnung zur Mahnung)
\newcommand{\mahnungsrechnungsformel} {hiermit stellen wir Ihnen zu o.g.\ Zahlungserinerung folgende Posten in Rechnung:}
\newcommand{\posten} {Posten}
\newcommand{\betrag} {Betrag}
\newcommand{\bitteZahlenBis} {Bitte begleichen Sie diese Forderung bis zum}

% anfrage (request_quotion)
\newcommand{\anfrage} {Quotation request}
\newcommand{\anfrageformel} {bitte nennen Sie uns für folgende Artikel Preis und Liefertermin:}
\newcommand{\anfrageBenoetigtBis} {Wir benötigen die Lieferung bis zum}  %Danach wird das Datum eingefügt, falls das grammatisch nicht funktionieren sollte müssen wir eine ausnahme für die sprache definieren
\newcommand{\anfragedanke} {Im Voraus besten Dank für Ihre Bemühungen.}

% bestellung/auftrag (purchase_order)
\newcommand{\bestellung} {Order}
\newcommand{\unsereBestellnummer} {Our order number}
\newcommand{\bestellformel} {we hereby order the following items:}

% einkaufslieferschein (purchase_delivery_order)
\newcommand{\einkaufslieferschein} {Eingangslieferschein}
}{
	\Ifstr{\lxlangcode}{DE}{
		\makeatletter
		\main@language{ngerman}
		\makeatother
		%%%%%%%%%%%%%%%%%%%%%%%%%%%%%%%%%%%%%%%%%%%%%%%
%standardphrasen und schnipsel in deutsch     %
%dient als vorlage für alle anderen sprachen  %
%%%%%%%%%%%%%%%%%%%%%%%%%%%%%%%%%%%%%%%%%%%%%%%


\newcommand{\anrede} {Sehr geehrte Damen und Herren,}
\newcommand{\anredefrau} {Sehr geehrte Frau}
\newcommand{\anredeherr} {Sehr geehrter Herr}


\newcommand{\nr} {Nr.}
\newcommand{\datum} {Datum}
\newcommand{\kundennummer} {Kunden-Nr.}
\newcommand{\ansprechpartner} {Ansprechpartner}
\newcommand{\bearbeiter} {Bearbeiter}
\newcommand{\gruesse} {Mit freundlichen Grüßen}
\newcommand{\vom} {vom}
\newcommand{\von} {von}
\newcommand{\seite} {Seite}
\newcommand{\uebertrag} {Übertrag}


\newcommand{\position} {Pos.}
\newcommand{\artikelnummer} {Art.-Nr.}
\newcommand{\bild} {Bild}
\newcommand{\keinbild} {kein Bild}
\newcommand{\menge} {Menge}
\newcommand{\bezeichnung} {Bezeichung}
\newcommand{\seriennummer}{Seriennummer}
\newcommand{\ean}{EAN}
\newcommand{\projektnummer}{Projektnummer}
\newcommand{\charge}{Charge}
\newcommand{\mhd}{MHD}
\newcommand{\einzelpreis} {E-Preis}
\newcommand{\gesamtpreis} {G-Preis}
\newcommand{\nettobetrag} {Nettobetrag}
\newcommand{\schlussbetrag} {Gesamtbetrag}

\newcommand{\weiteraufnaechsterseite} {weiter auf der nächsten Seite ...}

\newcommand{\zahlung} {Zahlungsbedingungen:}
\newcommand{\textTelefon} {Tel.:}
\newcommand{\textFax} {Fax:}

% angebot (sales_quotion)
\newcommand{\angebot} {Angebot}
\newcommand{\angebotsformel} {gerne unterbreiten wir Ihnen folgendes Angebot:}
\newcommand{\angebotdanke} {Wir danken für Ihre Anfrage und hoffen, Ihnen hiermit ein interessantes Angebot gemacht zu haben.}
\newcommand{\angebotgueltig} {Das Angebot ist freibleibend gültig bis zum}		%Danach wird das Datum eingefügt, falls das grammatisch nicht funktionieren sollte müssen wir eine ausnahme für die sprache definieren
\newcommand{\angebotfragen} {Sollten Sie noch Fragen oder Änderungswünsche haben, können Sie uns gerne jederzeit unter den unten genannten Telefonnummern oder E-Mail-Adressen kontaktieren.}
\newcommand{\angebotagb} {Bei der Durchführung des Auftrags gelten unsere AGB, die wir Ihnen gerne zuschicken.}


% auftragbestätigung (sales_order)
\newcommand{\auftragsbestaetigung} {Auftragsbestätigung}
\newcommand{\auftragsnummer} {Auftrag-Nr.}
\newcommand{\ihreBestellnummer} {Ihre Bestellnummer}
\newcommand{\auftragsformel} {hiermit bestätigen wir Ihnen folgende Bestellpostionen:}
\newcommand{\lieferungErfolgtAm} {Die Lieferung erfolgt am} %Danach wird das Datum eingefügt, falls das grammatisch nicht funktionieren sollte müssen wir eine ausnahme für die sprache definieren
\newcommand{\auftragpruefen} {Bitte kontrollieren Sie alle Positionen auf Übereinstimmung mit Ihrer Bestellung! Teilen Sie Abweichungen innerhalb von 3 Tagen mit!}
\newcommand{\proformarechnung} {Proforma Rechnung}

% lieferschein (sales_delivery_order)
\newcommand{\lieferschein} {Lieferschein}

% rechnung (invoice)
\newcommand{\rechnung} {Rechnung}
\newcommand{\rechnungsdatum} {Rechnungsdatum}
\newcommand{\ihrebestellung} {Ihr Bestellung}
\newcommand{\lieferdatum} {Lieferdatum}
\newcommand{\rechnungsformel} {für unsere Leistungen erlauben wir uns, folgende Positionen in Rechnung zu stellen:}
\newcommand{\zwischensumme} {Zwischensumme}
\newcommand{\leistungsdatumGleichRechnungsdatum} {Das Leistungsdatum entspricht, soweit nicht anders angegeben, dem Rechnungsdatum.}
\newcommand{\unserebankverbindung} {Unsere Bankverbindung}
\newcommand{\textKontonummer} {Kontonummer:}
\newcommand{\textBank} {bei der}
\newcommand{\textBankleitzahl} {BLZ:}
\newcommand{\textBic} {BIC:}
\newcommand{\textIban} {IBAN:}
\newcommand{\unsereustid} {Unsere USt-Identifikationsnummer lautet}
\newcommand{\ihreustid} {Ihre USt-Identifikationsnummer:}
\newcommand{\steuerfreiEU} {Steuerfreie, innergemeinschaftliche Lieferung.}
\newcommand{\steuerfreiAUS} {Steuerfreie Lieferung ins außereuropäische Ausland.}

\newcommand{\textUstid} {UStId:}

% gutschrift (credit_note)
\newcommand{\gutschrift} {Gutschrift}
\newcommand{\fuerRechnung} {für Rechnung}
\newcommand{\gutschriftformel} {wir erlauben uns, Ihnen folgenden Positionen gutzuschreiben:}

% sammelrechnung (statement)
\newcommand{\sammelrechnung} {Sammelrechnung}
\newcommand{\sammelrechnungsformel} {bitte nehmen Sie zur Kenntnis, dass folgende Rechnungen unbeglichen sind:}
\newcommand{\faellig} {Fälligkeit}
\newcommand{\aktuell} {aktuell}
\newcommand{\asDreissig} {30}
\newcommand{\asSechzig} {60}
\newcommand{\asNeunzig} {90+}

% zahlungserinnerung (Mahnung)
\newcommand{\mahnung} {Zahlungserinnerung}
\newcommand{\mahnungsformel} {man kann seine Augen nicht überall haben - offensichtlich haben Sie übersehen, die folgenden Rechnungen zu begleichen:}
\newcommand{\beruecksichtigtBis} {Zahlungseingänge sind nur berücksichtigt bis zum}
\newcommand{\schonGezahlt} {Sollten Sie zwischenzeitlich bezahlt haben, betrachten Sie diese Zahlungserinnerung  bitte als gegenstandslos.}

% zahlungserinnerung_invoice (Rechnung zur Mahnung)
\newcommand{\mahnungsrechnungsformel} {hiermit stellen wir Ihnen zu o.g. \mahnung \ folgende Posten in Rechnung:}
\newcommand{\posten} {Posten}
\newcommand{\betrag} {Betrag}
\newcommand{\bitteZahlenBis} {Bitte begleichen Sie diese Forderung bis zum}

% anfrage (request_quotion)
\newcommand{\anfrage} {Anfrage}
\newcommand{\anfrageformel} {bitte nennen Sie uns für folgende Artikel Preis und Liefertermin:}
\newcommand{\anfrageBenoetigtBis} {Wir benötigen die Lieferung bis zum}  %Danach wird das Datum eingefügt, falls das grammatisch nicht funktionieren sollte müssen wir eine ausnahme für die sprache definieren
\newcommand{\anfragedanke} {Im Voraus besten Dank für Ihre Bemühungen.}

% bestellung/auftrag (purchase_order)
\newcommand{\bestellung} {Bestellung}
\newcommand{\unsereBestellnummer} {Unsere Bestellnummer}
\newcommand{\bestellformel} {hiermit bestellen wir verbindlich folgende Positionen:}

% einkaufslieferschein (purchase_delivery_order)
\newcommand{\einkaufslieferschein} {Eingangslieferschein}
}{%%%%%%%%%%%%%%%%%%%%%%%%%%%%%%%%%%%%%%%%%%%%%%%
%standardphrasen und schnipsel in deutsch     %
%dient als vorlage für alle anderen sprachen  %
%%%%%%%%%%%%%%%%%%%%%%%%%%%%%%%%%%%%%%%%%%%%%%%


\newcommand{\anrede} {Sehr geehrte Damen und Herren,}
\newcommand{\anredefrau} {Sehr geehrte Frau}
\newcommand{\anredeherr} {Sehr geehrter Herr}


\newcommand{\nr} {Nr.}
\newcommand{\datum} {Datum}
\newcommand{\kundennummer} {Kunden-Nr.}
\newcommand{\ansprechpartner} {Ansprechpartner}
\newcommand{\bearbeiter} {Bearbeiter}
\newcommand{\gruesse} {Mit freundlichen Grüßen}
\newcommand{\vom} {vom}
\newcommand{\von} {von}
\newcommand{\seite} {Seite}
\newcommand{\uebertrag} {Übertrag}


\newcommand{\position} {Pos.}
\newcommand{\artikelnummer} {Art.-Nr.}
\newcommand{\bild} {Bild}
\newcommand{\keinbild} {kein Bild}
\newcommand{\menge} {Menge}
\newcommand{\bezeichnung} {Bezeichung}
\newcommand{\seriennummer}{Seriennummer}
\newcommand{\ean}{EAN}
\newcommand{\projektnummer}{Projektnummer}
\newcommand{\charge}{Charge}
\newcommand{\mhd}{MHD}
\newcommand{\einzelpreis} {E-Preis}
\newcommand{\gesamtpreis} {G-Preis}
\newcommand{\nettobetrag} {Nettobetrag}
\newcommand{\schlussbetrag} {Gesamtbetrag}

\newcommand{\weiteraufnaechsterseite} {weiter auf der nächsten Seite ...}

\newcommand{\zahlung} {Zahlungsbedingungen:}
\newcommand{\textTelefon} {Tel.:}
\newcommand{\textFax} {Fax:}

% angebot (sales_quotion)
\newcommand{\angebot} {Angebot}
\newcommand{\angebotsformel} {gerne unterbreiten wir Ihnen folgendes Angebot:}
\newcommand{\angebotdanke} {Wir danken für Ihre Anfrage und hoffen, Ihnen hiermit ein interessantes Angebot gemacht zu haben.}
\newcommand{\angebotgueltig} {Das Angebot ist freibleibend gültig bis zum}		%Danach wird das Datum eingefügt, falls das grammatisch nicht funktionieren sollte müssen wir eine ausnahme für die sprache definieren
\newcommand{\angebotfragen} {Sollten Sie noch Fragen oder Änderungswünsche haben, können Sie uns gerne jederzeit unter den unten genannten Telefonnummern oder E-Mail-Adressen kontaktieren.}
\newcommand{\angebotagb} {Bei der Durchführung des Auftrags gelten unsere AGB, die wir Ihnen gerne zuschicken.}


% auftragbestätigung (sales_order)
\newcommand{\auftragsbestaetigung} {Auftragsbestätigung}
\newcommand{\auftragsnummer} {Auftrag-Nr.}
\newcommand{\ihreBestellnummer} {Ihre Bestellnummer}
\newcommand{\auftragsformel} {hiermit bestätigen wir Ihnen folgende Bestellpostionen:}
\newcommand{\lieferungErfolgtAm} {Die Lieferung erfolgt am} %Danach wird das Datum eingefügt, falls das grammatisch nicht funktionieren sollte müssen wir eine ausnahme für die sprache definieren
\newcommand{\auftragpruefen} {Bitte kontrollieren Sie alle Positionen auf Übereinstimmung mit Ihrer Bestellung! Teilen Sie Abweichungen innerhalb von 3 Tagen mit!}
\newcommand{\proformarechnung} {Proforma Rechnung}

% lieferschein (sales_delivery_order)
\newcommand{\lieferschein} {Lieferschein}

% rechnung (invoice)
\newcommand{\rechnung} {Rechnung}
\newcommand{\rechnungsdatum} {Rechnungsdatum}
\newcommand{\ihrebestellung} {Ihr Bestellung}
\newcommand{\lieferdatum} {Lieferdatum}
\newcommand{\rechnungsformel} {für unsere Leistungen erlauben wir uns, folgende Positionen in Rechnung zu stellen:}
\newcommand{\zwischensumme} {Zwischensumme}
\newcommand{\leistungsdatumGleichRechnungsdatum} {Das Leistungsdatum entspricht, soweit nicht anders angegeben, dem Rechnungsdatum.}
\newcommand{\unserebankverbindung} {Unsere Bankverbindung}
\newcommand{\textKontonummer} {Kontonummer:}
\newcommand{\textBank} {bei der}
\newcommand{\textBankleitzahl} {BLZ:}
\newcommand{\textBic} {BIC:}
\newcommand{\textIban} {IBAN:}
\newcommand{\unsereustid} {Unsere USt-Identifikationsnummer lautet}
\newcommand{\ihreustid} {Ihre USt-Identifikationsnummer:}
\newcommand{\steuerfreiEU} {Steuerfreie, innergemeinschaftliche Lieferung.}
\newcommand{\steuerfreiAUS} {Steuerfreie Lieferung ins außereuropäische Ausland.}

\newcommand{\textUstid} {UStId:}

% gutschrift (credit_note)
\newcommand{\gutschrift} {Gutschrift}
\newcommand{\fuerRechnung} {für Rechnung}
\newcommand{\gutschriftformel} {wir erlauben uns, Ihnen folgenden Positionen gutzuschreiben:}

% sammelrechnung (statement)
\newcommand{\sammelrechnung} {Sammelrechnung}
\newcommand{\sammelrechnungsformel} {bitte nehmen Sie zur Kenntnis, dass folgende Rechnungen unbeglichen sind:}
\newcommand{\faellig} {Fälligkeit}
\newcommand{\aktuell} {aktuell}
\newcommand{\asDreissig} {30}
\newcommand{\asSechzig} {60}
\newcommand{\asNeunzig} {90+}

% zahlungserinnerung (Mahnung)
\newcommand{\mahnung} {Zahlungserinnerung}
\newcommand{\mahnungsformel} {man kann seine Augen nicht überall haben - offensichtlich haben Sie übersehen, die folgenden Rechnungen zu begleichen:}
\newcommand{\beruecksichtigtBis} {Zahlungseingänge sind nur berücksichtigt bis zum}
\newcommand{\schonGezahlt} {Sollten Sie zwischenzeitlich bezahlt haben, betrachten Sie diese Zahlungserinnerung  bitte als gegenstandslos.}

% zahlungserinnerung_invoice (Rechnung zur Mahnung)
\newcommand{\mahnungsrechnungsformel} {hiermit stellen wir Ihnen zu o.g. \mahnung \ folgende Posten in Rechnung:}
\newcommand{\posten} {Posten}
\newcommand{\betrag} {Betrag}
\newcommand{\bitteZahlenBis} {Bitte begleichen Sie diese Forderung bis zum}

% anfrage (request_quotion)
\newcommand{\anfrage} {Anfrage}
\newcommand{\anfrageformel} {bitte nennen Sie uns für folgende Artikel Preis und Liefertermin:}
\newcommand{\anfrageBenoetigtBis} {Wir benötigen die Lieferung bis zum}  %Danach wird das Datum eingefügt, falls das grammatisch nicht funktionieren sollte müssen wir eine ausnahme für die sprache definieren
\newcommand{\anfragedanke} {Im Voraus besten Dank für Ihre Bemühungen.}

% bestellung/auftrag (purchase_order)
\newcommand{\bestellung} {Bestellung}
\newcommand{\unsereBestellnummer} {Unsere Bestellnummer}
\newcommand{\bestellformel} {hiermit bestellen wir verbindlich folgende Positionen:}

% einkaufslieferschein (purchase_delivery_order)
\newcommand{\einkaufslieferschein} {Eingangslieferschein}
}
} % Ende EN


% Mandanten-/Firmenabhängigkeiten

% Pfad zu firmenspez. Angaben, sofern kein Unterordner mit dem Datenbanknamen des Mandanten in der Vorlage existiert, wird der Unterordner „firma“ verwendet. Der Datenbankname ist ab hier im Makro \identpath gespeichert
\setupIdentpath{\kivicompany}

% Lade die Konfiguration aus dem entsprechenden Unterordner
\newcommand{\telefon} {++49 228 92 98 2012}
\newcommand{\fax} {}
\newcommand{\firma} {Richardson \& Büren GmbH}
\newcommand{\strasse} {Römerstr. 45 - 47}
\newcommand{\ort} {53111 Bonn}
\newcommand{\ustid} {DE292363254}
\newcommand{\finanzamt} {Finanzamt Bonn-Innenstadt}
\newcommand{\email} {information@kivitendo-premium.de}
\newcommand{\homepage} {http://www.kivitendo-premium.de}



%Setze Briefkopf-logo falls vorhanden
\setkomavar{fromlogo}{\includegraphics[width=.25\linewidth]{\identpath/briefkopf}}

%Ganzseitiger Briefbogen als Hintergrund:
%\DeclareNewLayer[page,background,
%	contents={\includegraphics{Briefbogen}} %Hier muss der Dateinamen und ggf. die Bildgröße angepasst werden, falls es abweichende Maße vom Papierformat hat.
%]{background}
%\AddLayersToPageStyle{kivitendo.letter.first}{background}%Hintergrund für die erste Seite aktivieren
%\AddLayersToPageStyle{kivitendo.letter}{background}% Hintergrund für die übrigen Briefseiten aktivieren.


% Währungen/Konten
% Die Konfiguration bedindet sich in der Datei 
% \identpath/<euro/chf/usd/default>_account.tex
\setupCurrencyConfig{\identpath}{\lxcurrency}


% Befehl f. normale Schriftart und -größe

\KOMAoptions{
	fontsize=10pt,
	parskip=half-,% Absatzkennzeichnung durch Abstand statt Einzug
}
\renewcommand*{\familydefault}{\sfdefault}
\KOMAoptions{fontsize=10pt}

% Einstellungen f. Kopf und Fuss
\pagestyle{kivitendo.letter}
% Befehl f. laufende Kopfzeile:
% 1. Text f. Kunden- oder Lieferantennummer (oder leer, wenn diese nicht ausgegeben werden soll)
% 2. Kunden- oder Lieferantennummer (oder leer)
% 3. Belegname {oder leer}
% 4. Belegnummer {oder leer}
% 5. Belegdatum {oder leer}
% Beispiel: \ourhead{\kundennummer}{<%customernumber%>}{\angebot}{<%quonumber%>}{<%quodate%>}
\setkomafont{pagehead}{\scriptsize}
\newcommand{\ourhead}[5] {
\chead{
  \ifnum\thepage=1
  \else
      \makebox[\textwidth]{
      \Ifstr{#1}{}{}{#1: #2 \hspace{0.7cm}}
      #3
      \Ifstr{#4}{}{}{~\nr: #4}
      \Ifstr{#5}{}{}{\vom ~ #5}
      \hspace{0.7cm} - \seite ~ \thepage/\letterlastpage  ~-%
      }
  \fi
}
}


\normalfont % damit die footerbox schon in der standard-schriftart gebaut wird.
%% % Firmenfuss
% Das speichern als Box ermöglicht es, die Höhe automatisch anzupassen:
\setkomafont{pagefoot}{\tiny}

%Box generieren, um die Höhe des Fußres zu kennen
\newsavebox\footerbox
\begin{lrbox}\footerbox
	\usekomafont{pagefoot}%
     \begin{tabular*}{\textwidth}{@{\extracolsep{\fill}}p{5cm}p{4.5cm}lr@{}}%
	\firma                 & \email              & \textKontonummer & \kontonummer \\
	\strasse               & \homepage           & \textBank        & \bank \\
	\ort                   & \textUstid\ \ustid  & \textIban        & \iban \\
	\textTelefon~\telefon  & \finanzamt          & \textBic         & \bic \\
	\Ifstr{\fax}{}{}{\textFax~\fax} & &\textBankleitzahl	& \bankleitzahl
	\end{tabular*}
\end{lrbox}

%Fußhöhe auf Höhe der Box
\setlength{\footheight}{\dimexpr\ht\footerbox+\dp\footerbox}
\geometry{bottom=\dimexpr\csname g_kivi_margin_dim\endcsname +\footheight}
\savegeometry{kivi.letter@default}

%Box in den Fuß eintragen:
\cfoot{\usebox\footerbox}

\endinput
